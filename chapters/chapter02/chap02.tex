%
% File: chap02.tex
% Author: Wesley Hoffman
% Description: ACM Manual
%
\let\textcircled=\pgftextcircled
\chapter{ACM Manual}
\label{chap:manual}

\initial{T}he Student Chapters Program of the Association for Computing Machinery (ACM) was formed to introduce students to an educational and scientific association and to develop the habit of professional growth achieved through participation in local chapters. The purposes of the Student Chapters Manual are:
	\begin{itemize}
		\item to provide student chapter officers with a resource for answering 				questions about ACM and the benefits of student membership
      \item to describe the role and structure of a student chapter
		\item to explain the support and services available from ACM
		\item to offer suggestions for a range of activities possible for a student chapter
		\item to provide a single resource of useful information for the formation 				and operation of a student chapter
	\end{itemize}

Each ACM student chapter is issued a copy of the Manual. It is helpful for the faculty sponsor and all chapter officers to be familiar with its contents. Following each election, the outgoing chair will pass the Manual, along with other chapter records, to the new chair.

%=======
\section{History of the Manual}
\label{sec:sec01}

This Manual is the cumulative work of many people. The idea for the Student Chapter Manual originated with Dr. James R. Oliver, who chaired the Subcommittee on Student Chapters from 1961 to 1964, and who was subsequently named the first Chair of the ACM Committee on Student Membership and Chapters. He served until 1966, when Dr. Leland H. Williams accepted the position.
\\
\\
The subsequent chairs, Dr. Gary Carlson (1968-1970), Dr. Willard Gardner (1970-1972), Dr. Donald Hartford (1972-1974), and Dr. Barry J. Bateman (1974-1978), added refinements.
\\
\\
As you use this Manual in the formation and operation of a student chapter of the ACM, please pass along suggestions, comments, or ideas by writing to local\_activities@acm.org.

\section{The Association for Computing Machinery}
\label{sec:sec02}

The Association for Computing Machinery was founded in 1947 as the society for computing and information processing. Its purposes, as set forth in its Constitution, are as follows:
	\begin{itemize}
		\item To advance the sciences and arts of information processing including, but not restricted to, the study, design, development, construction, and application of modern machinery, computing techniques and appropriate languages for general information processing, for scientific computation, for the recognition storage, retrieval, and processing of data of all kinds, and for the automatic control and simulation of processes.
      \item To promote the free interchange of information about the sciences and arts of information processing both among specialists and among the public in the best scientific and professional tradition.
      \item To develop and maintain the integrity and competence of individuals engaged in the practice of the sciences and arts of information processing. ACM chapters exist as a means of achieving those purposes, and to serve the computing and information processing community, and the public, at the local level. Through technical meetings and lectures, chapters promote the free interchange of information among their members and the larger community in which they serve. Through workshops, chapters help develop and maintain the competence of computing and information processing personnel in their area. ACM chapters encourage membership and participation in the Association at its international level, and they also serve as a training ground for members who wish to become involved in the volunteer management of the Association.
	\end{itemize}

Chapters provide a range of activities and services including talks by local practitioners, visits from prominent speakers on the ACM Distinguished Lectureship Program circuit, technical and career workshops, field trips to computing installations, and social activities.

This handbook is intended as a general reference guide for officers of all student chapters of the Association for Computing Machinery (ACM). It is not intended as a model for the organization or operation of your local group. Each Chapter is encouraged to develop in the manner best suited to its own environment.

All chapter officers are volunteers and, when assuming office, are sometimes unsure how to accomplish the goals and fulfill the responsibilities of that office. This manual will help provide ideas to help all chapters operate more effectively, and to encourage participation of the entire chapter membership in the management and planning.

\section{Why Should a Student Join ACM?}
\label{sec:sec03}

There are a number of immediate benefits:
	\begin{itemize}
		\item Sharply reduced membership dues. Student membership is subsidized by the Association, resulting in a discount of up to 70\%.
      \item Reduced rates for subscription to all ACM publications.
		\item Special student membership rates in the ACM Special Interest Groups.
		\item Reduced rates on conference registration. Student members are eligible for low rates at national, regional, and topical meetings sponsored by the Association.
	\end{itemize}

Student membership in the ACM affords some less obvious benefits as well. Membership yields insight into computing as a science and a profession. Various publications and conferences keep students informed of state-of-the-art developments and allow them to meet and observe accomplished professionals from all areas of the industry. Participation in a professional organization is also prima-facie evidence of serious interest and dedication.

To find out more about membership in ACM, please visit the ACM student member website at http://www.acm.org/membership/student/student-toc

\section{Role of Student Chapters}
\label{sec:sec04}

ACM established student chapters to provide an opportunity for students to play a more active role in the Association and its professional activities. By encouraging organization of student chapters on college and university campuses, the Association is able to introduce students to the benefits of professional organizations. These benefits include regular meetings that encourage and enhance learning through exchange of ideas among students as well as between established professionals and students. Members of a student chapter may also take advantage of the activities and services provided by ACM, including the lectureship program, student programming and tutorial contests, and the publications program. Student chapters provide an obvious setting for the development and demonstration of leadership capabilities. Finally, students find the various activities of ACM and its student chapters both professionally and socially exciting as well as rewarding.

Initially, the ACM organizational structure was based solely on individual membership. In 1954, as a result of growth and the wishes of its members, chapters were officially formed. These chapters provided a means for people in a geographical area with a common interest in computing to exchange ideas and sponsor professional activities. Since the recognition of the Dallas-Fort Worth Chapter in 1954 as the first ACM chapter, the number of professional chapters has grown steadily to over 113 today, and they have become an integral part of the ACM organization. Student chapters were authorized by the ACM in 1961; the first was chartered at the University of Southwestern Louisiana.

Student chapters provide important services to ACM student members and offer a means whereby the ACM can provide scientific information on the industry to other members of the college or university community as well as to the general public. Moreover, professional chapters and student chapters are focal points for feedback from members to the ACM leadership. Finally, ACM Chapters are a training ground for the Association's future leaders. Over one-third of the current members of the ACM Council began their volunteer work with ACM as officers in the chapter system.

\section{Why Belong to an ACM Student Chapter?}
\label{sec:sec05}
In addition to the advantages of ACM membership listed in Section I, there are a number of benefits specifically available as a result of association with a student chapter.
\\
\textbf{Professional contacts:} Activities of student chapters, such as lecturer appearances and chapter participation at any of the numerous conferences, afford the opportunity to meet distinguished computing professionals.
\\
\textbf{Technical and professional growth:} Student chapter sponsorship of, or participation in, programs consisting of papers, debates, panel discussions and forums provide clear opportunity for augmented learning and development.
\\
\textbf{Development of leadership capabilities:} Opportunities for development and demonstration of leadership capabilities abound in the formation, growth, and sustenance of a student chapter. In addition to the various chapter offices, there are opportunities for chairing committees, conferences or symposia, organizing programming contests, coordinating professional development seminars, and leading panel discussions and round tables, to name just a few.
\\
\textbf{Career development:} The chapter can help its members select and prepare for a careers by creating a chapter newsletter, career-day programs, and graduate school forums. In addition, the student chapter can help locate and organize summer opportunities and internships by working with the department and career center to find and fill part-time and summer employment opportunities.
\\
\textbf{Tours:} Setting up a tour presents an opportunity to develop contacts as well as organizational skills.
\\
\textbf{Representation in the Association:} A student chapter acting as a group, through the Chapters Advisory Committee, may influence ACM activities or policy and, therefore, the profession.

\section{Organizational Structure of a Student Chapter}
\label{sec:sec06}

\section*{Officers}

The importance of leadership to the success of a student chapter cannot be overemphasized. Enthusiasm and dedication are the fundamental qualifications for both the student officers and the sponsor. In addition, each officer must be a student member in good standing of the ACM. Other requirements for chapter officers may be added in the student chapter's bylaws. A description of typical chapter officers, along with their respective duties, follows.
\subsection*{Chair}
The chair of any student chapter is the student leader responsible for all chapter activities. This person normally presides at all meetings of the chapter, and represents the chapter to the Association. However, all chapter members are encouraged to attend ACM meetings and become involved in the activities of the Association. Typically, the chair appoints all committees of the chapter.
\subsection*{Vice Chair}
The vice chair assumes the duties of the chair in the event of the chair's absence, and fulfills those duties assigned by the chair. Typically, the vice chair is chair of the program committee, which is responsible for arranging speakers, socials, and other activities for the chapter. Since the vice chair's duties can prove to be extensive, some chapters have chosen to create a separate office, the program chair, to be responsible for planning the various chapter activities. A variety of possible chapter activities is detailed in a later section of this manual.
\subsection*{Secretary}
The secretary keeps the minutes of all chapter meetings, prepares the annual chapter activity report, which is presented to the chapter at the end of the program year--normally the meeting when newly elected officers formally assume their posts, and prepares reports to be sent to ACM headquarters as required. The secretary must also notify ACM Headquarters of changes in officers or sponsor of the chapter. Likewise, the secretary must send official notification to the ACM Chapter Coordinator of any proposed changes in the Chapter\'s bylaws for approval, prior to their distribution to the membership.
\subsection*{Treasurer}
The responsibilities of the treasurer include collecting dues and maintaining the financial and membership records of the chapter. The treasurer must also file the annual report of the chapter's finances required by the Treasurer of ACM.
\subsection*{Faculty Sponsor}
The sponsor should assist, when necessary, with the programs and activities of the chapter to ensure adherence to the tenets of the Association, and to ensure responsible fiscal management of the Chapter's affairs. Hence, the characteristics of an effective sponsor are a sincere interest in the Student Chapter, and sensitivity to the needs of the Chapter, the Chapter officers, and membership.
\section*{KU Specific Officers}
\subsection*{Web Master}
The Web Master must maintain the ACM public facing website. Responsibilities
include developing and designing an appealing and professional forward-facing website, including information pertinent to the chapter, keeping the site\’s information up to date, and connecting this site to the various social media outlets and Rock Chalk Central.
\subsection*{Publicity Director}
The Publicity Director\'s primary directive is to create, maintain, and raise awareness of the Chapter in the University of Kansas and Lawrence community. In joint with the Chair, the PD will develop constructive relationships with individuals and organization outside of the Chapter. They are responsible for the maintainence of social media accounts, the Chapter's Rock Chalk Central, and photo library by posting frequent Chapter news, photos, events, and other pertinent information.
\subsection*{ESC Representative}
The Engineering Student Council (ESC) Representative is responsible for representing the Chapter at all council meetings. They are charged with delegating Student Senate and ESC information to the appropriate officers.  The ESC Rep is also the primary voting body for ACM in council meetings. Lastly, should the Chair not be able to attend mandatory \“Club President\” functions, the ESC Rep will attend in place.

\section*{SIG Officers}
Each SIG requires a Chair, Vice Chair and Secretary/Treasurer. These three officers are in full responsibility of SIG operations. As of January 2016, programming club is being developed as an official ACM@KU SIG so it is currently ran by two Co-Chairs.

\section{Elections}

The officers can make or break a student chapter. Finding the right people to assume leadership roles is essential to the chapter's success. Thus, the election of officers should be taken seriously. We suggest the following procedures for the nomination and election of a chapter's officers.
\subsection*{Nominating Committee}
The Committee should be composed of chapter members in good standing who have a sincere interest in the continued success of the chapter. Members of the committee may be current officers or members-at-large of the chapter. To allow adequate time for the Nominating Committee to select nominees, the chapter chair should appoint the Nominating Committee at least one month prior to the annual election meeting. The nominating committee is responsible for presenting at least one nominee for each of the chapter offices prior to the annual election meeting.

The nominating committee should meet and select a proposed set of candidates for each office, and then contact each prospective candidate to determine whether that person will be willing and able to serve, if elected. The chosen slate of candidates for each office should then be presented to the general membership and additional candidates may be nominated, either at a meeting or by petition, prior to the election meeting. The final slate of nominees, along with any supporting statements, is then promptly sent to all voting members of the student chapter for their consideration prior to the election meeting.
\subsection*{Election Meeting}
A meeting for the election of officers should be held annually in accordance with the provisions of the student chapter bylaws. It is best to hold the election early enough so the newly elected officers will have time to work with the outgoing officers. An annual report on the chapter's activities during the past year and the state of the treasury are provided to the members at this meeting by the outgoing officers.
\subsection*{Continuity}
The most persistent problem facing student chapters is the rise and fall of chapter activity due to the transience of the membership and officers. Careful selection of a chapter sponsor can go some distance toward alleviating this problem. In addition, current officers need to continually "groom" many possible successors through task assignments and active involvement in the operation of the chapter.

Elections should be timed to include an overlap period -- that is, we suggest you have the election meeting early enough in the spring to allow elections of new officers prior to the expiration of the terms of the current officers, allowing for an effective and smooth transition. Similarly, if the sponsor resigns, see the department chair to arrange for selection of a replacement in time to allow for overlap if at all possible.

\section{Committees}
\label{sec:committees}

Student chapter officers usually delegate at least some tasks to committees. The chapter may have one or more standing committees responsible for continuing or recurring tasks. Examples include a program committee, a membership committee, a fundraising committee, a luncheon committee, and an awards committee. In addition, ad hoc committees can be created for specific one time events such as an ad hoc committee for a regional conference or bylaws revisions. As stated earlier in this section, committees are usually appointed by the chapter chair, as specified in the chapter bylaws.

\subsection*{Membership Committee}
A membership committee should be appointed and chaired by an enthusiastic and conscientious member. This committee should have a supply of promotional materials, including ACM student member brochures and applications. It is recommended that at least once a year, this committee plan a major membership drive. This drive could include an ACM table at some major chapter event; direct contact of majors in computer science and related fields, and students registered in computer science courses; mailings to other student members of the ACM in the chapter area--labels are available from ACM headquarters; or hosting a welcoming get-together for new or interested students at the beginning of the academic year.
\subsection*{Intramural Programming Competition Committee}
An IPC committee should be appointed and chaired each year to develop problems, solicit funds from sponsers, purchase prizes, and organize the competition specifics. Each year the Chapter holds a programming competition in the Spring for the students of the University of Kansas. The programming competition is held in Eaton Hall for several hours on a Saturday, and prizes are distributed to the winners in each category.
\subsection*{Engineering Expo Committee}
An Engineering Expo committee should be appointed and chaired each year to develop activities and create decorations based on the theme of Expo that year. Activities should be developed with the age of visitors in mind (K-12).

\section{Membership}

Members are the life of the chapter; without an active membership, there can be no student chapter. This is especially significant to student chapters since, by definition, the present and prospective members are transient. Therefore, a continuing membership recruitment program is essential for success of the chapter.

Perhaps the most effective means of membership recruiting is an excited current membership. If current members are enthusiastic, this enthusiasm is bound to spill over to fellow students, who can be encouraged to explore the potential benefits of student chapter affiliation.

An interesting chapter is a successful chapter. Therefore, the key to success is to stimulate and maintain interest. The purpose of this section is to present some guidelines and suggestions for maintenance of a viable and stimulating student chapter.

The chapter must have regularly scheduled activities to remain viable. The vice chair or a separate officer can serve as program chair. Selection, encouragement, and support of this officer is vital -- certainly one of the keys to success. The other is encouraging the active involvement of the membership. Full member participation is essential because (1) there is simply too much work for the officers alone, and (2) active and expected involvement makes each member feel important and provides a sense of accomplishment. There are a number of ways to involve members in chapter operations, including ad hoc committees, the nominating committee, a membership committee, or an arrangements committee (for a chapter trip to a regional or national conference, or for a chapter-hosted event such as a lecture or film). This kind of activity structure allows for further delegation of such tasks as publicity (posters, news media), brochures, campus tours, reception arrangements, and dinner arrangements, and therefore promotes the active involvement of a significant number of chapter members.
\\
\\
"I think the best leaders are the lazy ones. They get everyone enthused but let others do all the work. Funny thing is the more you delegate, the more everybody likes it! Everyone gets to be in charge of something like the newsletter, the April Meeting, the web site, etc. At that point, the leader's job is to encourage his hatchlings to lazy up a little too so there's room for even more volunteers to take care of mailing lists, snacks at the meetings, etc. There can never be too many small jobs that a new volunteer can handle with their eyes closed."

"All these people will bond with each other, pretty soon the chapter will take on a life of its own simply because the folks like to socialize with each other."

-- Bob Lamm, Boston ACM SIGGRAPH

\section{Financing Chapter Activities}
\label{sec:sec07}
A successful student chapter requires resources -- financial, material, and service support for its programs and member services. In this section, some of those resources will be discussed; suggestions on how to get them will be offered.

The student chapter will likely need computer resources for maintaining its databases of membership and mailing lists. It will need access to photocopying equipment, telephone service, meeting space, and space for storing files of chapter records, correspondence, publicity materials, application forms and other chapter property.

The chapter will need financial resources to cover local arrangements for ACM lecturers, for refreshments at meetings or special events, for publications (printing and distribution of a student chapter newsletter, for example), for awards sponsored by the chapter, for travel to conferences, and for other expenses in conjunction with your Chapter programs.

\subsection{Almost all of the funding to support these activities must be found locally}
One source of possible funding is your school's student organization office. Most campuses allow student groups to petition the institution, often through its student government, for a share of the student activity moneys. Recognition and funding by the institution may require modifications in the Student Chapter Bylaws to accommodate school restrictions. Please keep in mind that bylaws changes must be reviewed and approved by ACM prior to adoption by the chapter.

Support is often available from academic departments. This may take the form of secretarial services, support for necessary phone calls, and even partial support of local expenses for an ACM lecturer, especially if the department was consulted concerning the choice of the speaker and scheduling of his/her visit. In many instances, cooperative support is also available from the campus or department computer center. Possibilities here include an account email alias for your chapter, and permission to use the message of the day to announce various student chapter events and activities.

An obvious source of revenue is chapter membership dues. These dues should be kept reasonable, since chapter members are already paying Association dues. Your chapter can decide what is reasonable, but we suggest you make an effort to consult as many members as possible before determining the yearly dues.

One of the most common sources of funding for a student chapter is from chapter fund raising activities. The possibilities here are limited only by the imagination of the chapter members. Examples of successful fund raising projects at student chapters include the following:
	\begin{itemize}
		\item Bakeouts and sales of t-shirts in and around the dormitories
		\item Sales of computer produced art
		\item Holding a local technical meeting
	\end{itemize}

\subsection{Student Chapter Responsibilities}
This manual makes clear that ACM considers student chapters an important and integral part of its operation. In addition to the privileges for both student members and their institutions inherent in an ACM student chapter charter, however, the chapter incurs certain responsibilities. These responsibilities include:
	\begin{itemize}
		\item To further the purposes of the Association in a responsible way
		\item To promote ACM membership
		\item To educate others about the value and importance of ACM
		\item To encourage active involvement in the ACM by your college
		\item To foster the professional development of the membership
		\item To plan and execute an active and regular program of events
	\end{itemize}
In particular, to assure its continued existence and support from the Association, every student chapter must meet certain minimum. Minimum requirements include:
	\begin{itemize}
		\item Maintenance of student membership of not less than 3 (whom are the officers)
		\item Sponsorship of at least two general chapter functions per year
		\item Submission of at least one Student Chapter Activity Report each year
		\item Records
	\end{itemize}
The chapter must maintain financial, membership, and correspondence records. The financial records should include dues records of members, the expenses and income of each chapter fundraiser and the expense of each chapter activity. The membership records should include accurate school addresses, telephone numbers, and email addresses, ACM membership numbers, and special related interests of members. The correspondence records of the chapter should include chapter correspondence to and from speakers, ACM headquarters, Membership Activities Board Chapter records should also include copies of the chapter bylaws, the letter of petition to ACM, the letter of chapter recognition by ACM, and the chapter charter certificate. These records should be kept in a place easily accessible to all the chapter officers. A chapter office is ideal; if your chapter does not have an office, you may be able to arrange for file space in the department or program office -- ask your chapter sponsor to help!

\subsection{Financial Responsibilities of ACM Chapters}
Overview of Policies
\\
\\
All ACM chapters collect and disburse their own funds. ACM Bylaw 9, Sections 4 and 5, and ACM Bylaw 6, Section 5 provide the most complete information regarding the rights and responsibilities of ACM chapters regarding financial matters. A summary of those rights and responsibilities is as follows:
	\begin{enumerate}
		\item Any chapter that collects, holds or disburses funds on behalf of the Association or any of its branches must submit an annual accounting of such funds.
Chapter funds will be accounted for in the Association's IRS return unless the chapter specifically desires to file a separate return with the IRS. If a chapter chooses to file its own return, it must provide a copy of the return to the Treasurer.
		\item Failure to submit financial reports is grounds for revocation of charter as provided in the Constitution and Bylaws of the Association.
        \item Any chapter desiring to solicit funds of more than \$5,000 cash or equivalent goods or services from a single source must obtain approval from the ACM in advance.
        \item  Solicitations of donations of small magnitude for a specific goal of a specified time do not require presidential approval. Contact the Local Activities Program Director with questions.
        \item Disbursements of funds for those expenditures necessary for the normal operation of the chapter do not require presidential approval. Any chapter desiring to disburse funds beyond those necessary for normal operations must obtain advance approval of the President of the Association.
        \item Upon dissolution of a chapter or revocation of a chapter's charter, all assets of the chapter become the property of the Association. The only exception to this rule is in the case of certain student chapters, whose educational institutions require that such assets be transferred to them for a purpose within the contemplation of section 501 (c) of the Internal Revenue Code of 1954.
	\end{enumerate}

\subsection{Statement of Cash Receipts and Disbursements}
The Statement of Cash Receipts and Disbursements must be prepared annually by each chapter. A copy of the form is located in the appendix of this handbook.

As shown on this form, the chapter's name, address and federal employer identification number is requested, along with an accounting of all cash receipts and cash disbursements for the ACM fiscal year (from July 1 to June 30).

To aid in completing the form, it is suggested that each chapter maintain its detailed financial records to conform to the classifications of cash receipts and cash disbursements shown on the form.

The purpose of the Statement of Cash Receipts and Disbursements is twofold; to make ACM aware of the financial activity of each chapter on an annual basis, and to enable ACM Headquarters to file the Group Form 990 Return with the IRS. The specific authorization for including a particular chapter in the ACM Group Return is made on the authorization form sent along with the Statement of Cash Receipts and Disbursements. On the authorization form, the treasurer and chairperson of the chapter either authorize ACM to include the chapter with the Group Return, or state that the chapter will be filing a Form 990 Return with the IRS independently of ACM. ACM encourages all chapters to file with the ACM Group Return.

\subsection{Chapter's Relationship to the IRS}
501 (c) (3) Status
\\
\\
ACM is exempt from federal income taxes under Section 501 (c) (3) of the Internal Revenue Code of 1954 since the Association is organized and operates exclusively for scientific and educational purposes. This exemption was granted the ACM on March 21, 1957 (see copy of IRS determination letter in the appendix). Although ACM is not required to file income tax returns, the Association is required to file an annual informational return (Form 990). The Association also files Group Form 990 Returns for chapters. ACM is also required to file an annual Form 990-T Return with the IRS for any unrelated business income.
\\
\\
The IRS returns are due on or before November 15 of each year, covering information for the prior fiscal year ended June 30. The financial data reported to ACM Headquarters by each chapter on the Statement of Cash Receipts and Disbursements must be compiled and summarized. An organization that fails to file a Form 990 Return with the IRS by the due date can be charged with a penalty. It is therefore essential that chapters respond to ACM Headquarters by August 30 to enable ACM to file the Group 990 Return by November 15th.
\\
\\
Group Exemption:
\\
\\
ACM has a Group Exemption Number (number 1931) which recognizes the ACM chapters as sharing in our 501 (c) (3) exemption status, and thus recognizing them as exempt from all federal income taxes. To maintain this group exemption, ACM Headquarters is required to send an annual update to the IRS with information regarding changes in chapter addresses and chairpersons, and a list of newly chartered or non-chartered chapters.
\\
\\
Sales Tax Exemption:
\\
\\
ACM has been granted an exemption from New York State local sales tax. ACM chapters have the same right to be exempt from State and local sales taxes in those states and localities which grant such exemptions; however, each chapter must apply to the authorities in the state in which they operate. Chapters should contact their state department of revenue for the necessary forms, and should contact the Director of Finance at ACM Headquarters with any questions.
\\
\\
Investigate getting state tax exemption and/or postage permit where applicable. A quarterly update on sales tax exemption information for all states and Canadian provinces is available.
\\
\\
Federal Employer Identification Number:
\\
\\
ACM has a Federal Employer Identification Number (EIN), which is similar to an individual's social security number. This number is required by the IRS for purposes of filing informational or any other type of return with the IRS. Each ACM chapter is required to have an EIN as well. Each year, ACM Headquarters applies for EINs on behalf of the newly chartered chapters and chapters that have not been assigned EINs. Headquarters will notify the chapters of their EINs when they are received. The chapter's EIN should be used when opening bank accounts, and should be noted on the annual financial report filed with ACM Headquarters.
\\
\\
Responding to IRS Requests:
\\
\\
The IRS may send your chapter blank tax returns each year. Since most chapters choose to be included in the ACM Group 990 Return, these forms should not be prepared. The IRS may also send chapters reminder notices, and in some cases, delinquent notices. If your chapter receives such a notice, send it to the Local Activities program Director at ACM Headquarters, who will have the ACM Director of Finance respond to the IRS.


%=========================================================
