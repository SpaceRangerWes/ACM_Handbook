%
% File: chap04.tex
% Author: Wesley Hoffman
% Description: SILC Nonsense
%
\let\textcircled=\pgftextcircled
\chapter{Dealing with SILC}
\label{chap:silc}

\initial{T}he Student Involvement and Leadership Center (SILC) "is responsible for coordinating registered university organizations and providing leadership education experiences for students in addition to providing programs and services to specific target populations including fraternity/sorority members, non-traditional students, and students of all gender identities, gender expressions and sexual orientations." What this entails for the Chapter is that we need to be up to date and in good regards with SILC to be considered a KU student group. This is not a necessity required by ACM, but SILC is where much of our funding and outreach is sourced.

%======================

\section{Rock Chalk Central}
\label{sec:sec01}

Rock Chalk Central is the social media net for KU student organizations. This is where you will be registering the Chapter each year with SILC, adding new members to the Chapter, assigning officer titles, and more. It is important to keep the Chapter's RCC page up-to-date because it is what the University will use to judge our activity. This can affect funding, awards, and recognition.
There are several responsibilities on RCC that will be taken care of by different officers:
\begin{itemize}
	\item Publicity Director: Responsible for keeping news, photos, and events up-to-date. Events have to be approved by SILC so submit these well enough ahead of the event itself. News and events are formatted in the style of Markdown so become familiar with that text standard.
	\item Web Master: Responsible for the cover photos. Have fun and be creative.
	\item Secretary: Responsible for sending and approving membership requests, removing members, keeping the officer list up-to-date, and uploading and keeping the forms and documents folders clean. They will be responsible for all else that is not under the charge of the Publicity Director and Web Master.
\end{itemize}
As a standing rule, the Chair is to be Primary Contact for ACM@KU. If they would like to deligate this to a different officer, that is their prerogative.

\section{Registering}
\label{sec:sec02}

All organizations functioning on the Lawrence Campus of the University of Kansas are required to register with the University. Registration provides several advantages; official identification as a University affiliated group, use of designated University facilities and services, coordination and communication of group activities with campus administration and other organizations, and eligibility to receive funding from Student Senate.
\\
Organizations must be established for legal purposes consistent with the broad educational aims of the University and in accord with regulations, guidelines, and policies of the University, the City of Lawrence, and the State of Kansas. However, registration does not imply University endorsement of the purposes of an organization, nor does the University assume sponsorship of or responsibility for any group activities on or off University property.
\\
Should a student organization not follow general guidelines that are consistent with University policies, the group may be subject to losing their status as a student organization with the Student Involvement \& Leadership Center. If a student organization loses its registration, the president may appeal to the Office of the Vice Provost for Student Affairs in order to regain status as an organization.

\subsection{Organization Classifications}

\subsection*{Student Organizations:} A Student Organization is composed primarily of currently enrolled students \– 75\% of the membership or more. All officers must be currently enrolled students.
\subsection*{Campus Organizations:} A campus organization is composed primarily of members from the University community \– 75\% of the members are currently enrolled students, KU staff members and/or their spouses. The officers of the organization must be current full-time faculty/staff members of the University community.
\subsection*{Community Organizations:} A Community Organization is an organization whose stated purpose benefits both the Lawrence and University communities. Community organizations are encouraged to register with the University if their programs and services are beneficial to members of the University community.

\subsection{Additional Registration Information}

Online Registration Information. An organization must re-register its group each year with the Student Involvement and Leadership Center. To re-register your group, login to Rock Chalk Central. In order to login, you must have a personal KU user id and password.

If you were an officer/advisor for the previous year, you will see the groups you were affiliated with, once you login.
If you were not an officer/advisor for the previous year, you will not see any groups listed when you login. [I am not sure what happens after this]

Policy Statement on Orgs-l List:  Student organizations may use the ORGS-L listserv to notify members of events (such as but not limited to meetings, fundraisers, speeches, rallies or protests) that are organized, co-sponsored and/or supported by student groups registered with SILC.  SILC's policy strictly prohibits profanity and derogatory statements.  In addition, the ORGS-L listserv will not be used to discuss, debate or offer opinion, and/or the promotion of a political candidates. Violation of this policy will cause the individual to be removed for the remainder of the year.

Student and Advisor Electronic Signatures: When officers and the advisor submit the online organization application, each agrees to abide by the Regents Policy on Organization Membership and the Equal Opportunity Statement of the University of Kansas as stated at the bottom of this page. You can also view additional student group information by going to the Student Organizations Website. There you will find mission statements, contact information, and website addresses for each group on campus.

\subsection{Registration Procedures}

Organizations wishing to register with the University of Kansas must meet the following requirements:
\begin{itemize}
	\item Provide the University with the name and a statement of purpose for the organization.
	\item Be nonprofit in nature.
	\item Annually renew the organization registration (beginning July 1st each year).
	\item Maintain in the Student Involvement and Leadership Center a current list of names and email addresses of officers, advisers, and/or liaison person where applicable.
	\item Upon request, provide a copy of the organization's charter, constitution, or by-laws, including those of organizations outside the University with which the group is affiliated.
	\item A minimum of three members is required to register an organization with the Student Involvement and Leadership Center.
	\item Have an adviser who is a member of the current faculty or professional staff of the University, or approved by the Student Involvement and Leadership Center if the selected adviser is not on staff.
	\item Confirm the Electronic Registration. By doing so, the group is acknowledging they will adhere to all applicable Regents and University regulations affecting registered organizations and, in particular, the Regents and University Policies on Nondiscrimination in Organizational Membership.
\end{itemize}
It is the practice of the Student Involvement and Leadership Center to publish contact information for each registered organization. This information will be listed on the Student Organization directory page so interested individuals have a means of contacting the organization. Officers/adviser have the option of listing their telephone number and/or email address with the online directory or not having that information published at all. That decision is made by the group member submitting the registration application to SILC. At a minimum, a group email address should be listed so individuals interested in the group can contact them.

Applications for registration may be submitted online to the Student Involvement and Leadership Center any time during the year. Groups will have until September 15th of each year to submit updated information to the SILC office. If at that time the group has not submitted a re-registration application, the group will no longer be registered with SILC, and will need to submit a new student group registration form online. SILC will review the organization's registration materials and determine (1) the group's eligibility to register with the University of Kansas, and, (2) the organization's category for registration purposes. The responsibility for verification of membership rests with the organization and advisor. Registration status is granted administratively by the Student Involvement and Leadership Center. The organization's president and advisor will be notified via email of the organization registration status.

Groups wishing to appeal a decision about registration status should submit a written letter to the Director of the Student Involvement \& Leadership Center. The petition will be reviewed by three representatives of the Student Senate Executive Committee appointed by the President of the student body. The committee will make a recommendation to the Director of the Student Involvement \& Leadership Center.

The Student Involvement and Leadership Center maintains complete files of registration materials throughout the fiscal year in which the registration is active. A record of the University of Kansas organization registration will by maintained for historical purposes.

\subsection{Benefits to Registered Organizations}

Registering with the University entitles organizations to a number of benefits. The current available facilities and services are listed below by category. These facilities and services will be periodically reviewed, and organizations will be advised of any revisions.
\begin{enumerate}
	\item \textbf{Use of the University name in the organization's title:} (Student and Campus) \\
However, the University cannot permit its image (name) to be used in any commercial announcement, commercial or artistic production, or in any other context where endorsement of a product, organization, person, or cause is explicitly or implicitly conveyed.
	\item \textbf{Request student activity fee funding:} (Student, Campus, Community) \\
Student and campus groups may request funding through the established Student Senate funding process; community groups may request consideration of a contractual arrangement with the Student Senate.
	\item \textbf{Use of University facilities:} (Student, Campus)
	\begin{itemize}
		\item Schedule meeting rooms, public auditoriums and other designated spaces. Request office space, work station space, locker space, and/or mailbox in the student organizations area in the Kansas Union.
		\item Schedule the Information Booth located on Jayhawk Boulevard for information and/or sales purposes.
	\end{itemize}
	\item \textbf{Use of University Services:} (Student, Campus and Community)
	\begin{itemize}
		\item Inclusion in university publications, including the Student, Faculty and Staff directory.
		\item Establish an email and/or web page account with Academic Computing Services.
		\item Publicize activities through University Events Committee. Request a table for the Fall \& Spring semester Organization Information Fairs.
		\item Receive information materials from the Student Involvement and Leadership Center regarding University procedures, policies, and activities.
	\end{itemize}
	\item \textbf{Use of University Services:} (Student and Campus)
	\begin{itemize}
		\item Establish a checking account at no charge with the Comptroller's Office.
		\item Utilize university staff and programming resources.
		\item Use of campus mail for official business of the organization in accordance with established university and state policies.
		\item Utilize the Student Senate Advertising Program, receive student organizational advertising rates from the University Daily Kansan and KJHK radio station.
		\item Receive food services discount rates on self-service refreshments from the Kansas and Burge Unions for use at the Unions only.
	\end{itemize}
\end{enumerate}


\subsection{Discrimination Policy of the University of Kansas}

The University of Kansas prohibits discrimination on the basis of race, color, ethnicity, religion, sex, national origin, age, ancestry, disability, status as a veteran, sexual orientation, marital status, parental status, gender identity, gender expression and genetic information in the University\’s programs and activities. The following person has been designated to handle inquiries regarding the non-discrimination policies:
\\
\\
Director of the Office of Institutional Opportunity and Access, IOA@ku.edu
1246 W. Campus Road, Room 153A, Lawrence, KS 66045, (785) 864-6414
\\
\\
TDD (785)864-2620, TTY 711

\subsection{Regents Policy on Nondiscrimination in Organizational Membership}

Policy and Procedures Manual Chapter II: Policies and Procedures, Section E: Students, Item 6: Student Organizations and Activities.
\\
\\
\textbf{a.} The established policy of the Board of Regents of the State of Kansas prohibits discrimination on the basis of age, race, color, religion, sex, marital status, national origin, physical handicap or disability, status as a Vietnam Era Veteran, sexual orientation or other factors which cannot be lawfully considered, within the institutions under its jurisdiction. All fraternal and campus-related organizations shall follow this policy in the selection of their members, except the prohibition against sex discrimination shall not apply to social fraternities or sororities which are excluded from the application of Title IX of the Education Amendments of 1972 (20 U.S.C. Sec. 1681 et seq.). (6-27-02)
\\
\\
\textbf{b.} The right of organizations to establish standards for membership is acknowledged, provided that all students are afforded equal opportunity to meet those standards. Just as all students have the right to choose those with whom they would associate on the campus, an organization shall have the right to select its members subject to these principles. Nothing in this policy shall be interpreted as imposing a requirement which would violate the principle of selection on the basis of individual merit.
\\
\\
\textbf{c.} The responsibility for compliance with this policy lies with each organization. In discharge of this responsibility, each organization shall acknowledge its understanding of this policy. Such acknowledgement shall assure that there exist no restrictions on membership, either local or national, which violate this policy. (9-24-65; 1-21-77)
\\
\\
\textbf{d.} Determination of Appropriate Student Activities and Fees: The chief executive officer shall be ultimately responsible for reviewing proposed expenditures from fees required of every student as a condition of enrollment and as determining whether such expenditures are in support of an appropriate student activity.



%======================
